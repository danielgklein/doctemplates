% Options for packages loaded elsewhere
\PassOptionsToPackage{unicode}{hyperref}
\PassOptionsToPackage{hyphens}{url}
%
\documentclass[
  11pt,
  a4paperpaper,
]{article}
\usepackage{amsmath,amssymb}
\usepackage{iftex}
\ifPDFTeX
  \usepackage[T1]{fontenc}
  \usepackage[utf8]{inputenc}
  \usepackage{textcomp} % provide euro and other symbols
\else % if luatex or xetex
  \usepackage{unicode-math} % this also loads fontspec
  \defaultfontfeatures{Scale=MatchLowercase}
  \defaultfontfeatures[\rmfamily]{Ligatures=TeX,Scale=1}
\fi
\usepackage{lmodern}
\ifPDFTeX\else
  % xetex/luatex font selection
\fi
% Use upquote if available, for straight quotes in verbatim environments
\IfFileExists{upquote.sty}{\usepackage{upquote}}{}
\IfFileExists{microtype.sty}{% use microtype if available
  \usepackage[]{microtype}
  \UseMicrotypeSet[protrusion]{basicmath} % disable protrusion for tt fonts
}{}
\makeatletter
\@ifundefined{KOMAClassName}{% if non-KOMA class
  \IfFileExists{parskip.sty}{%
    \usepackage{parskip}
  }{% else
    \setlength{\parindent}{0pt}
    \setlength{\parskip}{6pt plus 2pt minus 1pt}}
}{% if KOMA class
  \KOMAoptions{parskip=half}}
\makeatother
\usepackage{xcolor}
\usepackage[margin=1in]{geometry}
\usepackage{graphicx}
\makeatletter
\newsavebox\pandoc@box
\newcommand*\pandocbounded[1]{% scales image to fit in text height/width
  \sbox\pandoc@box{#1}%
  \Gscale@div\@tempa{\textheight}{\dimexpr\ht\pandoc@box+\dp\pandoc@box\relax}%
  \Gscale@div\@tempb{\linewidth}{\wd\pandoc@box}%
  \ifdim\@tempb\p@<\@tempa\p@\let\@tempa\@tempb\fi% select the smaller of both
  \ifdim\@tempa\p@<\p@\scalebox{\@tempa}{\usebox\pandoc@box}%
  \else\usebox{\pandoc@box}%
  \fi%
}
% Set default figure placement to htbp
\def\fps@figure{htbp}
\makeatother
\setlength{\emergencystretch}{3em} % prevent overfull lines
\providecommand{\tightlist}{%
  \setlength{\itemsep}{0pt}\setlength{\parskip}{0pt}}
\setcounter{secnumdepth}{5}
% definitions for citeproc citations
\NewDocumentCommand\citeproctext{}{}
\NewDocumentCommand\citeproc{mm}{%
  \begingroup\def\citeproctext{#2}\cite{#1}\endgroup}
\makeatletter
 % allow citations to break across lines
 \let\@cite@ofmt\@firstofone
 % avoid brackets around text for \cite:
 \def\@biblabel#1{}
 \def\@cite#1#2{{#1\if@tempswa , #2\fi}}
\makeatother
\newlength{\cslhangindent}
\setlength{\cslhangindent}{1.5em}
\newlength{\csllabelwidth}
\setlength{\csllabelwidth}{3em}
\newenvironment{CSLReferences}[2] % #1 hanging-indent, #2 entry-spacing
 {\begin{list}{}{%
  \setlength{\itemindent}{0pt}
  \setlength{\leftmargin}{0pt}
  \setlength{\parsep}{0pt}
  % turn on hanging indent if param 1 is 1
  \ifodd #1
   \setlength{\leftmargin}{\cslhangindent}
   \setlength{\itemindent}{-1\cslhangindent}
  \fi
  % set entry spacing
  \setlength{\itemsep}{#2\baselineskip}}}
 {\end{list}}
\usepackage{calc}
\newcommand{\CSLBlock}[1]{\hfill\break\parbox[t]{\linewidth}{\strut\ignorespaces#1\strut}}
\newcommand{\CSLLeftMargin}[1]{\parbox[t]{\csllabelwidth}{\strut#1\strut}}
\newcommand{\CSLRightInline}[1]{\parbox[t]{\linewidth - \csllabelwidth}{\strut#1\strut}}
\newcommand{\CSLIndent}[1]{\hspace{\cslhangindent}#1}
\usepackage[symbol]{footmisc}
\usepackage{setspace}
\usepackage{indentfirst}
\usepackage{bookmark}
\IfFileExists{xurl.sty}{\usepackage{xurl}}{} % add URL line breaks if available
\urlstyle{same}
\hypersetup{
  hidelinks,
  pdfcreator={LaTeX via pandoc}}

\author{}
\date{\vspace{-2.5em}}

\begin{document}

\begin{titlepage}
\thispagestyle{empty}

\begin{center}
\vspace*{3cm}

{\LARGE Paper Title\footnote{Acknowledgments, IRB status, etc.} \par}

\vspace{2cm}

{\large Your Name\footnote{Position, Department, University. Email: you@email.com} \par}

\vspace{0.5cm}

{\large Department, University \par}

\vspace{1.5cm}

{\large August 30, 2025 \par}
\vspace{1.5cm}

\begin{minipage}{0.9\textwidth}
\textbf{Abstract:}  
Type your abstract here.

\vspace{0.5cm}

Word Count: 366
\end{minipage}

\end{center}
\end{titlepage}

\newpage

\doublespacing
\setlength{\parindent}{15pt}

This will be your introduction. There is no section header, because
generally introductions should not be labeled ``Introduction''---it's at
the start, and it's telling us what we're reading, we already know we're
being introduced!

\section*{Literature Review}

Literature gets reviewed here. Arguments get made. Xie, Allaire, and
Grolemund (2018) Lorem ipsum dolor sit amet consectetur adipiscing elit.
Quisque faucibus ex sapien vitae pellentesque sem placerat. In id cursus
mi pretium tellus duis convallis. Tempus leo eu aenean sed diam urna
tempor. Pulvinar vivamus fringilla lacus nec metus bibendum egestas.
Iaculis massa nisl malesuada lacinia integer nunc posuere. Ut hendrerit
semper vel class aptent taciti sociosqu. Ad litora torquent per conubia
nostra inceptos himenaeos.

\section*{Methods}

Methods are described here. Lorem ipsum dolor sit amet consectetur
adipiscing elit. Quisque faucibus ex sapien vitae pellentesque sem
placerat. In id cursus mi pretium tellus duis convallis. Tempus leo eu
aenean sed diam urna tempor. Pulvinar vivamus fringilla lacus nec metus
bibendum egestas. Iaculis massa nisl malesuada lacinia integer nunc
posuere. Ut hendrerit semper vel class aptent taciti sociosqu. Ad litora
torquent per conubia nostra inceptos himenaeos.

\section*{Results}

Yay! Your research has results! Describe them here. Lorem ipsum dolor
sit amet consectetur adipiscing elit. Quisque faucibus ex sapien vitae
pellentesque sem placerat. In id cursus mi pretium tellus duis
convallis. Tempus leo eu aenean sed diam urna tempor. Pulvinar vivamus
fringilla lacus nec metus bibendum egestas. Iaculis massa nisl malesuada
lacinia integer nunc posuere. Ut hendrerit semper vel class aptent
taciti sociosqu. Ad litora torquent per conubia nostra inceptos
himenaeos.

\section*{Conclusion}

Conclude your paper here. Tempus leo eu aenean sed diam urna tempor.
Pulvinar vivamus fringilla lacus nec metus bibendum egestas. Iaculis
massa nisl malesuada lacinia integer nunc posuere. Ut hendrerit semper
vel class aptent taciti sociosqu. Ad litora torquent per conubia nostra
inceptos himenaeos.

\newpage

\section*{References}

\phantomsection\label{refs}
\begin{CSLReferences}{1}{0}
\bibitem[\citeproctext]{ref-xieMarkdownDefinitiveGuide2018}
Xie, Yihui, J.J. Allaire, and Garrett Grolemund. 2018. \emph{R Markdown:
The Definitive Guide}. Boca Raton, Florida: Chapman and Hall/CRC.
\url{https://bookdown.org/yihui/rmarkdown}.

\end{CSLReferences}

\end{document}
